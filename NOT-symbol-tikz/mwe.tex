% https://tex.stackexchange.com/questions/737348/what-is-the-best-practice-to-declare-a-tikzpicture-as-a-math-symbol
\documentclass{article}
\usepackage{tikz,amsmath}
\usetikzlibrary{calc}
\begin{document}
\makeatletter
\def\pgfgetnodeheight(#1)#2{
  \path ($(#1.south) - (#1.north)$);
  \pgfmathsetmacro#2{veclen(\pgf@x, \pgf@y)}
  \edef#2{#2pt}
}
\def\pgfgetnodewidth(#1)#2{
  \path ($(#1.east) - (#1.west)$);
  \pgfmathsetmacro#2{veclen(\pgf@x, \pgf@y)}
  \edef#2{#2pt}
}
\makeatother
\DeclareRobustCommand{\notleftrightarrow}{%
    \mathrel{\begin{tikzpicture}[baseline=-.5ex]%
        \node[inner sep=0pt,outer sep=0pt] (a) {$\Longleftrightarrow$};
        \pgfgetnodewidth(a)\nWidth
        \pgfgetnodeheight(a)\nHeight
        \draw[line width=.5pt](-.6*\nHeight,-.25*\nWidth)--(.6*\nHeight,.25*\nWidth);
      \end{tikzpicture}%
    }\kern-1em%%
}
\[
\sum_{k}^{n}f(k)%
\fbox{%
\begin{tikzpicture}[baseline=-.5ex]%
\fbox{%
  \node[inner sep=0pt,draw] (a) {$\Longleftrightarrow$};%
  \pgfgetnodewidth(a)\nWidth%
  \pgfgetnodeheight(a)\nHeight%
  \fbox{%
    \draw[line width=.5pt](-.6*\nHeight,-.25*\nWidth)--(.6*\nHeight,.25*\nWidth);
  }
  \fill[red] (a) circle[radius=1pt];
}
\end{tikzpicture}}%\mkern-15mu%
ABC
\]

\[
\sum_{k}^{n}f(k) \notleftrightarrow ABC
\]
\end{document}