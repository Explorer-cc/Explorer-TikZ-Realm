\documentclass{ctexart}
\usepackage{geometry}   %%调整页面间距
\geometry{a4paper,left=1.25in,right=1.25in,top=1in,bottom=1in} 
\usepackage{hyperref}
\usepackage{url}
\usepackage{xcolor}
\usepackage{listings}
\lstdefinestyle{lfonts}{
  basicstyle   = \footnotesize\ttfamily,
  stringstyle  = \color{purple},
  keywordstyle = \color{blue!60!black}\bfseries,
  commentstyle = \color{olive}\scshape,
}
\lstdefinestyle{lnumbers}{
  numbers     = left,
  numberstyle = \tiny,
  numbersep   = 1em,
  firstnumber = 1,
  stepnumber  = 1,
}
\lstdefinestyle{llayout}{
  breaklines       = true,
  tabsize          = 2,
  columns          = flexible,
}
\lstdefinestyle{lgeometry}{
  xleftmargin      = 20pt,
  xrightmargin     = 0pt,
  frame            = tb,
  framesep         = \fboxsep,
  framexleftmargin = 20pt,
  backgroundcolor = \color{blue!10},  %%控制背景颜色
}
\lstdefinestyle{lgeneral}{
  style = lfonts,
  style = lnumbers,
  style = llayout,
  style = lgeometry,
}
\lstdefinestyle{python}{
    language = {Python},
    style    = lgeneral,
}

\begin{document}
\textbf{example.2} 使用listings自定义代码环境\par
具体而言,需要使用lstdefinestyle定义样式的不同细节,详细的配置可参见texdoc listings。 \par
上述配置单来自 \url{https://www.zhihu.com/question/65508676}。\par
使用\textbf{lstinline}命令插入行内代码 \par
\lstinline[style = python]|print('To explore the world,you should explore yourself first!')|

使用\textbf{lstlisting}环境插入行间代码 \par
\begin{lstlisting}[style = python]
def build(self, number):
    for _ in range(number):
        t = random.uniform(0, 2 * pi)  
        x, y = heart_function(t)
        self._points.add((x, y))

    for _x, _y in list(self._points):
        for _ in range(3):
            x, y = scatter_inside(_x, _y, 0.05)
            self._edge_diffusion_points.add((x, y))

    point_list = list(self._points)
    for _ in range(4000):
        x, y = random.choice(point_list)
        x, y = scatter_inside(x, y, 0.17)
        self._center_diffusion_points.add((x, y))
\end{lstlisting}

\end{document}