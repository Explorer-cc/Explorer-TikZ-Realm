\documentclass{ctexart}
\usepackage[many]{tcolorbox}
\usepackage{fontawesome}
\usepackage{xcolor}
\usepackage{listings}
\usepackage{url}
\usepackage{hyperref}
\usepackage{geometry}   %%调整页面间距
\geometry{a4paper,left=1.25in,right=1.25in,top=1in,bottom=1in} 
\lstdefinestyle{Matlab}{
    language        =   Matlab, % 选择对应的代码语言对应高亮关键字
    frame           =   none,
    basicstyle      =   {\small\color{white}\tt},
    mathescape,
    emphstyle       =   {\small\bfseries\color{white}},
    numbers         =   none,
    stepnumber      =   2,
    numbersep       =   3em,
    numberstyle     =   \small\ttfamily\bfseries\tiny,
    keywordstyle    =   \small\ttfamily\bfseries\color{orange},
    stringstyle     =   \small\ttfamily\bfseries\color{white},
    breaklines      =   true,   % 自动换行,
    columns         =   fixed,  %字间距不固定会很丑
    basewidth       =   .5em,
    commentstyle    =   \small\ttfamily\bfseries\color{cyan!30},
    backgroundcolor =   \color{macosbox@bgdark},
    tabsize         =   4,
    showspaces      =   false,
    showstringspaces=   false,
    morekeywords    =   {Out,In,maketitle,pip,matplotlib,pandas,numpy,np,dtype},
}
\definecolor{macosbox@bgdark}{RGB}{30,30,30}
\definecolor{macosbox@borddark}{RGB}{30,30,30}
\definecolor{macosbox@titledark}{RGB}{61,62,64}
\definecolor{macosbox@linedark}{RGB}{116,116,118}
\newtcolorbox{macboxd}[2][]{%自定义maxbox环境并配置对应参数(我的建议是直接chao)
    arc             =   5pt,
    enhanced,
    %colupper        =   white,
    colupper        =   black,
    coltitle        =   white,
    %colback         =   macosbox@bgdark,
    colback         =   white,
    boxrule         =   0mm,
    frame style     =   {draw=macosbox@linedark,fill=macosbox@borddark},
    title style     =   {color=macosbox@titledark},
    drop fuzzy shadow=darkgray,
    title           ={\hspace{-1em}\textcolor[RGB]{220, 105, 98}{\scriptsize\faCircle}\ \ \textcolor[RGB]{244, 191, 79}{\scriptsize\faCircle}\ \ \textcolor[RGB]{97, 197, 84}{\scriptsize\faCircle}\rightline{#2\hspace{0.5\textwidth}}},#1
}
\begin{document}
为了追求更美观的视觉效果,参照\textbf{carbon} \quad\url{https://carbon.now.sh/}这一macbox的样式配置了对应的minted样式,可以实现较好的代码展示效果(如下)。\par
在此也特别感谢\LaTeX{} Studio \quad \url{https://www.latexstudio.net/} 交流群大神群友 \textbf{西安-雨霓-1210} 提供的莫大帮助~

\begin{macboxd}{Matlab}
    \begin{lstlisting}[style=Matlab]
%% 一维特征LSTM网络训练
numFeatures = 1;   %特征为一维
numResponses = 1;  %输出也是一维
numHiddenUnits = 200;   %创建LSTM回归网络,指定LSTM层的隐含单元个数200。

layers = [ ...
            sequenceInputLayer(numFeatures)    %输入层
            lstmLayer(numHiddenUnits)  % LSTM层
            fullyConnectedLayer(numResponses)    %为全连接层
            regressionLayer];      %其计算回归问题的半均方误差模块
%指定训练选项,求解器设置为adam, 1000轮训练。
%梯度阈值设置为1。指定初始学习率0.01,在125轮训练后通过乘以因子0.2来降低学习率
options = trainingOptions('adam', ...
    'MaxEpochs',1000, ...
    'GradientThreshold',1, ...
    'InitialLearnRate',0.01, ...      
    'LearnRateSchedule','piecewise', ...
    'LearnRateDropPeriod',400, ...      
    'LearnRateDropFactor',0.15, ...     
    'Verbose',0,  ...  %如果将其设置为true,则有关训练进度的信息将被打印到命令窗口
    'Plots','training-progress');    
net = trainNetwork(XTrain,YTrain,layers,options);
    \end{lstlisting}
\end{macboxd}

\end{document}