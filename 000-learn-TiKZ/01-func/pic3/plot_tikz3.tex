\documentclass[border=5pt]{standalone}
\usepackage{tikz}
\usepackage{pgfplots}
\usetikzlibrary{intersections}
\usetikzlibrary{math}
\pgfplotsset{compat=1.18}
\usepackage{amsmath}
\tikzset{elegant/.style={smooth,thick,samples=50}}
% the syntax \ShowIntersection[<projections>]{<path 1>}{<path 2>}{<color>}
\NewDocumentCommand{\ShowIntersection}{ ommm }{
    \filldraw 
    [name intersections={of=#2 and #3, name=i, total=\t}] 
    [#4, opacity=1, every node/.style={above left, black, opacity=1}] 
    \foreach \s in {1,...,\t}{(i-\s) circle (1pt)}
    \IfValueT{#1}{
        \foreach \s / \nodecontent / \nodexshift in {#1}{
            (i-\s) edge[#4, dashed] (i-\s |- 0,0) (i-\s |- 0,0) node
            [color = #4, below=1em, anchor=base, font=\footnotesize, xshift=\nodexshift]
            {$\nodecontent$}
        }
    };
}
%%%
% - \ShowIntersection{A}{D}{orange} works as before,
% - \ShowIntersection[1/x_1, 2/x_2]{A}{D}{orange} marks the first intersection with x_1 and the second with x_2 (so you can use A and B if you need),
% - \ShowIntersection[2/x]{A}{D}{orange} marks only the second intersection.
%%%

\begin{document}
    \begin{tikzpicture}
        \begin{axis}[
            axis x line = middle,
            axis y line = middle,
            xmin = -1.5,
            xmax = 4.5,
            ymin = -1.2,
            ymax = 3.2
        ]
        \node[below left=2.5pt] at (0,0) {$O$};
        \addplot[elegant,domain=-4:4,name path global=A]{(x-1)*ln(x)-x+1} node [left=2pt] {$f(x)$};
        \addplot[blue,mark=*,mark size=1pt] coordinates {(1,0)};
        \addplot[blue,mark=*,mark size=1pt] coordinates {(e,0)};
        \addplot[blue,thick,name path global=B]{-x+1};
        \addplot[blue,thick,name path global=C]{(1-1/e)*x-e+1};
        \addplot[orange,thick,name path global=D] coordinates {(-1,0.5) (4,0.5)};
        \ShowIntersection[1/x_1/2pt, 2/x_2/-2pt]{A}{D}{orange}
        \ShowIntersection[1/x'_1/-2pt]{B}{D}{blue}
        \ShowIntersection[1/x'_2/2pt]{C}{D}{blue}
        \end{axis}
    \end{tikzpicture}
\end{document}

% https://tex.stackexchange.com/questions/720331/what-is-the-best-practice-to-define-a-newcommand-to-derive-the-projection-of-the/720343#720343