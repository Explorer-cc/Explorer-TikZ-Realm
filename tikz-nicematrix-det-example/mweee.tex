% https://ask.latexstudio.net/ask/question/17826.html
\documentclass{article}
\usepackage{newpxmath}
\usepackage{nicematrix}
\usepackage{tikz}
\usetikzlibrary{calc}
\begin{document}
\pagestyle{empty}
\setlength{\extrarowheight}{1.5mm}
\ExplSyntaxOn
\def\N{8}
\def\ratio{\fpeval{\N/(2*(\N-1))}}
\[
\begin{vNiceMatrix}
    % \c_math_subscript_token
    \int_step_inline:nn {\N} {%
        a\c_math_subscript_token{#11}%
        \int_step_inline:nnn {2}{\N} {%
            & a\c_math_subscript_token{#1{##1}}%
        }%
        \int_compare:nNnTF {#1} < {\N} {\\} {}%
    }%
    \CodeAfter%
    \begin{tikzpicture}[line~cap=round]
        % 定义主/副对角线的向量
        \coordinate (main-vec) at ($(\N-\N.center)-(1-1.center)$);
        \coordinate (secondary-vec) at ($(\N-1.center)-(1-\N.center)$);
        % 处理副对角线方向的其他 \N-1 条虚线
        \draw[orange,dashed,thick] (1-\N.center) -- (\N-1.center);
        \int_step_inline:nn {\N-1} {
            \draw[orange,dashed,thick,rounded~corners=5mm] (1-#1.center) --++(secondary-vec) --++($\ratio*(main-vec)$) -- (\int_eval:n{#1+1}-\N.center);
        }
        % 处理主对角线方向的其他 \N-1 条实线
        \draw[magenta,thick] (1-1.center) -- (\N-\N.center);
        \int_step_inline:nn {\N-1} {
            \draw[magenta,thick,rounded~corners=5mm] (1-\int_eval:n{#1+1}.center) --++(main-vec) --++($\ratio*(secondary-vec)$) -- (\int_eval:n{\N-#1+1}-1.center);
        }
    \end{tikzpicture}
    \end{vNiceMatrix}
\]
\ExplSyntaxOff 
\end{document}